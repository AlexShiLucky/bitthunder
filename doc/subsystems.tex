\section{Sub Systems}
BitThunder can be separated into a number of sub-systems. There are some base sub-systems,
which all other sub-systems use for allocating and managing resources.

This document covers the sub-systems required for version 1.0.0 of BitThunder.

\begin{tabular}{|l|l|r|}
\hline 
Subsystem & Description & Completeness \\ 
\hline 
Handle Manager & Tracks all resources via handles, ensuring cleanup is possible. & 90\% \\ 
\hline 
Device Manager & Provides abstracted access to devices and their resources. & 50\% \\ 
\hline 
Process Manager & Fundamental process abstraction, allows clean killing. & 75\% \\ 
\hline
Real-Time Scheduler & Provides real-time slicing of the CPU resource. & 100\% \\ 
\hline 
Volume Manager & Manages volumes and their partitions, based on block device media. & 0\% \\ 
\hline
Filesystem Manager & Abstracts file-system implementations. & 50\% \\ 
\hline 
Network Stack & Manages network device and network APIs. & 25\% \\ 
\hline 
\end{tabular} 

\subsection{Handle Manager}
Provides management of process / kernel level handles. Tracks their resources, and provides a novel cleanup
mechanism.

The BT_HANDLE type is a fundamental type in BitThunder. Almost all resources have an associated BT_HANDLE in some
way. 

\subsubsection{Data Structures}
\includegraphics{figures/architecture/handles.1}


\subsection{Device Manager}
Provides several "directories" for finding and enumerating abstracted devices.
\begin{itemize}
\item BT_MACHINE_DESCRIPTION
\item BT_INTEGRATED_DEVICE
\end{itemize}

The device manager, is one of the most complex aspects of BitThunder...
\subsubsection{Machine Descriptions}
\subsubsection{Integrated Devices}
\subsubsection{GPIO}
\subsubsection{Interrupts}

\subsection{Memory Manager}
Provides APIs for allocating and freeing memory. To begin with a simple implementation is provided
but this should be replaced later, as the requirement for using protected / virtual memory becomes real.

\subsection{Process Manager}

\subsection{Real-Time Scheduler}
This is the core of the process manager, providing the fundamental time-slicing of the CPU resource.


\subsection{Block Device Manager}

\subsection{Volume Manager}

\subsection{Filesystem Manager}

\subsection{Network Stack}


